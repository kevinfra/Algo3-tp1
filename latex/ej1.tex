\section{Ejercicio 1}
    % 1. Describir detalladamente el problema a resolver dando ejemplos del mismo y sus soluciones.
    \subsection{Descripción del problema}
		\begin{figure}[ht]
			\begin{center}
				\includegraphics[width=0.5\columnwidth]{imagenes/kaioken.jpg}
				\caption{Los orígenes del CrossFit}
			\end{center}
		\end{figure}
        Gokú debe entrenar un ejército de $N$ guerreros para enfrentarse a Majin Boo. Para lograr ese objetivo, desea organizar peleas entre sus luchadores. En cada pelea, los guerreros serán divididos en dos bandos que pelearán entre sí. El objetivo es que, para cualquier par de luchadores, haya al menos una pelea en la que se encuentren en bandos distintos. Al mismo tiempo, el número total de peleas debe ser el mínimo necesario.

        La resolución del problema consiste en elaborar un programa que recibe como entrada el valor de $N$, y devuelve la cantidad mínima de peleas que deben llevarse a cabo para cumplir el cometido de Gokú y a continuación, por cada una de las peleas, una línea indicando el bando al que pertenece cada uno de los luchadores durante la misma. De haber más de una solución posible, puede devolverse cualquiera de ellas.

        Por ejemplo, si el programa recibe como entrada el valor \texttt{8}, cualquiera de las siguientes tres salidas sería correcta:

        \begin{center}\begin{tabular}{l @{\hskip 2em} | @{\hskip 2em} l @{\hskip 2em} | @{\hskip 2em} l}
            \texttt{3}               & \texttt{3}  &              \texttt{3}               \\
            \texttt{1 1 1 1 2 2 2 2} & \texttt{1 1 2 2 2 1 2 1} & \texttt{2 1 1 1 2 1 2 2} \\
            \texttt{1 1 2 2 1 1 2 2} & \texttt{1 1 2 1 2 2 1 2} & \texttt{1 2 1 1 2 2 2 1} \\
            \texttt{1 2 1 2 1 2 1 2} & \texttt{2 1 2 1 1 1 2 2} & \texttt{1 1 2 1 2 2 1 2} \\
        \end{tabular}\end{center}

    % 2. Explicar de forma clara, sencilla, estructurada y concisa, las ideas desarrolladas para la resolución del problema. Utilizar pseudocódigo y lenguaje coloquial (no código fuente). Justificar por qué el procedimiento resuelve efectivamente el problema.
    \subsection{Solución propuesta}
        La solución que se propone para el problema resulta más fácil de comprender con un sencillo cambio de óptica. En lugar de entender una solución como una sucesión de peleas, donde a cada una de ellas le corresponde una lista representando a un peleador, puede pensarse en una asignación entre cada guerrero y su \emph{historial de peleas}: la lista de los bandos que le fueron asignados en cada una de las peleas. Por ejemplo, si el historial de peleas de un guerrero es \texttt{2122}, esto significa que en la primera, la tercera y la cuarta pelea formó parte del bando número 2, mientras que en el segundo enfrentamiento peleó en el bando número 1. En una solución presentada según el formato requerido, los historiales de peleas de cada guerrero pueden verse si se la lee por columnas.

        Para que una solución al problema sea válida, es necesario cualquier par de peleadores diferentes se enfrente al menos en una pelea. Puede observarse que esto equivale a que el historial de peleas asignado a cada guerrero sea distinto, es decir, que difiera como mínimo en un dígito de todos los demás. Por otro lado, se pide que se lleve a cabo el mínimo número de peleas necesario, o lo que es lo mismo, que la longitud de los historiales de peleas sea la menor posible.

        Observando el problema desde esta perspectiva, es posible apreciar su semejanza con la construcción de un sistema de numeración posicional. Mientras que en un sistema de numeración se busca representar a los números naturales mediante una serie finita de símbolos provenientes de un alfabeto también finito, en una solución al problema planteado se busca asignar a cada peleador un historial de peleas, que consiste también en una serie finita de elementos de un conjunto finito: los bandos $1$ y $2$. Además, en un sistema de numeración, la representación de cada número tiene, en cierto sentido, la mínima longitud posible: estas se van asignando en orden creciente, y si queda ``libre'' una representación de cinco dígitos para un número, se utiliza esta antes que una de seis. Esto se parece al requisito de optimalidad requerido por el problema. Por último, dados dos números distintos, es necesario que sus representaciones en un sistema de numeración difieran como mínimo en un dígito, de forma análoga a lo que se desea que suceda con los historiales de peleas de dos guerreros diferentes.

        El sistema de numeración binario utiliza, comúnmente, los símbolos \texttt{0} y \texttt{1}. Sin embargo, esta elección es completamente arbitraria. Si se piensa en una variante de este sistema construida en base a los símbolos \texttt{1} y \texttt{2}, el problema de asignar un historial de peleas para $N$ guerreros que cumpla los requerimientos de Gokú puede reducirse a encontrar una representación en dicho sistema para los primeros $N$ números naturales. (A modo ilustrativo, pueden observarse las columnas de las soluciones de ejemplo expuestas en la descripción del problema. Si se reemplazan los símbolos $\lbrace\mathtt{1},\mathtt{2}\rbrace$ por $\lbrace\mathtt{0},\mathtt{1}\rbrace$, las columnas corresponden exactamente a la escritura binaria de los números entre $0$ y $7$).

        El algoritmo propuesto para resolver el problema planteado se basa en esta idea, y consiste simplemente en escribir los números naturales entre $0$ y $N-1$ en notación binaria utilizando $\lceil \log_2(N) \rceil$ dígitos para cada uno de ellos, pero cambiando los símbolos \texttt{0} y \texttt{1} por \texttt{1} y \texttt{2}, respectivamente. Estas escrituras representarán el historial de peleas de cada guerrero. Dado que para escribir los primeros $N$ naturales en notación binaria son necesarios exactamente $\lceil\log_2(N)\rceil$ dígitos, esta será la mínima cantidad de peleas necesarias para que todos los guerreros se enfrenten alguna vez entre sí. La línea correspondiente a la $i$-ésima pelea se obtiene tomando en orden los $i$-ésimos dígitos de las escrituras generadas; en otras palabras, leyendo las mismas ``por columnas''.

        El resultado así obtenido cumple con los requisitos necesarios para ser solución del problema. En efecto:
        \begin{itemize}
            \item Dados dos guerreros distintos, estos se enfrentan como mínimo en una pelea, ya que sus historiales de peleas (las escrituras binarias de sus respectivos índices) difieren al menos en un elemento.

            \item No existe una solución con menos peleas. Para ver esto, es importante notar que, dado cualquier $k \in \mathbb{N}$, se pueden generar como máximo $2^k$ escrituras binarias distintas de longitud $k$. En otras palabras, si Gokú hace que más de $2^k$ luchadores se enfrenten en $k$ peleas, habrá al menos dos de ellos que compartirán bando en todas las peleas.

            Consideremos, para cualquier cantidad de guerreros $N$, una serie de $k$ peleas, donde $k \leq \lceil \log_2(N) \rceil - 1$. En tal caso
            \[ 2^k \leq 2^{\lceil \log_2(N) \rceil - 1} = \frac{1}{2} 2^{\lceil \log_2(N) \rceil} < N\]
            dado que $ 2^{\lceil \log_2(N) \rceil}$ es la mínima potencia de dos mayor o igual que $N$. Como acaba de verse, esto significa que habrá al menos un par de guerreros que no se enfrentará nunca. Es decir, no existen soluciones válidas al problema con menos de $\lceil \log_2(N) \rceil$ peleas.
        \end{itemize}

        \subsubsection{Detalles implementativos}
            El algoritmo fue implementado en lenguaje C++. Para almacenar la solución, se recurre a la clase \texttt{vector}, proporcionada por la librería estándar del lenguaje.

            La ejecución consiste en un ciclo que se repite $\lceil \log_2(N) \rceil$ veces. La $i$-ésima iteración de este ciclo corresponde a la $i$-ésima pelea de la solución, y durante ella se construye la línea correspondiente a la misma, escribiendo repetidamente los símbolos \texttt{1} y \texttt{2}, alternando entre ellos cada $2^i$ símbolos. Es decir, la primera línea generada tiene la forma \texttt{1212121...}, la segunda, \texttt{1122112...}, y así sucesivamente. Si se observan las columnas del resultado así obtenido, se encuentran (con sus dígitos en orden inverso) las escrituras binarias de los primeros $N$ naturales, escritas con los símbolos \texttt{1} y \texttt{2}.

        \begin{codesnippet}
        \begin{verbatim}
cantidadPeleas = parte entera de log_2(n)
peleas = vector de tamaño cantidadPeleas inicializado en 2
saltos = 1
para i entre 0 y cantidadPeleas:
    para j entre 0 y n, con incrementos de 2 * saltos:
        para k entre 0 y saltos:
            si j + k < n:
                peleas[i][j + k] = 1
            fin si
        fin para
    fin para

    saltos = 2 * saltos
fin para

devolver peleas
        \end{verbatim}
        \end{codesnippet}

            En rigor, dado que la matriz necesaria para almacenar la solución es creada al comienzo de la ejecución, se aprovecha este momento para inicializarla con el símbolo \texttt{2}, utilizando el ciclo recién explicado para reemplazarlo por el símbolo \texttt{1} donde es necesario.

            % ¿MANDAMOS PSEUDOCÓDIGO? - NO, pero igual te quiero fran :) 

    % 3. Deducir una cota de complejidad temporal del algoritmo propuesto y justificar por qué el algoritmo cumple la cota dada. Utilizar el modelo uniforme.
    \subsection{Complejidad teórica}
         
      Durante la ejecución del algoritmo se utiliza una matriz con $\lceil \log_2(N) \rceil$ filas y $N$ columnas. En primer lugar, se inicializa esta matriz en \texttt{2}, lo cual tiene un costo del orden de $N \times \log(N)$. Para generar las peleas se recorre esta matriz; se tiene un ciclo que recorre las filas, y dentro del mismo, dos ciclos anidados que se encargan de recorrer las columnas. Los límites de estos últimos dos ciclos varían según la fila; en la fila $i$-ésima, el ciclo externo itera hasta $N$ con incrementos de tamaño $2^i$ ($\frac{N}{2^i}$ iteraciones), mientras que el ciclo interno itera hasta $2^i$ con saltos de a $1$ ($2^i$ iteraciones).

      Por lo tanto, la cantidad de veces que se ejecuta el ciclo interno, sumando sobre todas las filas de la matriz, es

      \[\sum_{i = 0}^{\log(N)} \sum_{j = 0}^{\frac{N}{2^i}} 2^i = \sum_{i = 0}^{\log(N)} \frac{N}{2^i} 2^i = \sum_{i = 0}^{log(N)} N = N \times \log(N)\]

      Como dentro de este último ciclo solo se modifica el valor de una posición de la matriz, con un costo $\ord(1)$, se puede concluir que la complejidad total del algoritmo es $\ord(N \times \log(N))$. 


    % 4. Dar un código fuente claro que implemente la solución propuesta. Se deben incluir las partes relevantes del código como apéndice del informe impreso entregado.

    % 5. Realizar una experimentación computacional para medir la performance del programa implementado. Usar un conjunto de casos de test en función de los parámetros de entrada, con instancias aleatorias e instancias particulares (de peor/mejor caso en tiempo de ejecución, por ejemplo). Presentar en forma gráfica una comparación entre los tiempos medidos y la complejidad teórica calculada y extraer conclusiones.
    \subsection{Experimentación}

	A la hora de medir el rendimiento del algoritmo y confirmar la complejidad
	teórica se tuvieron que considerar varios aspectos tales que los
	resultados de la experimentación reflejaran lo mejor posible su
	comportamiento.

	Dado que el problema consiste en una única entrada, los casos de prueba
	posibles se reducen al conjunto de los números enteros positivos. Como la
	complejidad temporal del algoritmo es $\theta(Nlog(N))$, no hay un peor
	y mejor escenario, por lo tanto no es necesario establecer condiciones sobre
	la entrada.

	El experimento consiste en la medición de los tiempos de ejecución para
	entradas cada vez más grandes. A continuación se describen las condiciones aplicadas
	para el mismo:

	\begin{itemize}
		\item{Como entrada se utilizó $N = k$ con $200 < k < 100000$}
		\item{Por cada $N$, se repitió 40 veces la medición y se calculó un
			promedio.}
	\end{itemize}

	De estos puntos, cabe destacar la elección de comenzar con $N = 200$. Esta
	decisión surge del hecho de que en valores pequeños las mediciones de tiempo
	son más sensibles a perturbaciones del sistema en el que están corriendo.
	Por lo tanto, en base a las pruebas realizadas, se optó por ignorar los
	primeros valores dado que no aportaban información pertinente al experimento.

	Los resultados obtenidos fueron los siguientes. En los gráficos, $T$ representa el
    tiempo de ejecución obtenido en las mediciones.

	\newcommand\constante{3}
	\begin{figure}[H]
		\centering
		\caption{}
		\label{fig:exp1:tiempo_base}
		\begin{tikzpicture}
			\begin{axis}[
					title={},
					xlabel={Tamaño de entrada ($N$)},
					ylabel={Tiempo de ejecución (nanosegundos)},
					scaled x ticks=false,
					scaled y ticks=false,
					enlargelimits=0.05,
					width=0.5\textwidth,
					height=0.5\textwidth,
					legend pos=south east,
					legend cell align=left,
					xmin=1
				]
				\addplot[color=black] table[x index=0,y index=1]{../exp/kaioKenOutput};
				\addplot[color=red] table[x index=0, y expr={x*ln(x)*\constante}]{../exp/kaioKenOutput};
				\legend{$T$, $c*N*log(N)$}
			\end{axis}
		\end{tikzpicture}
	\end{figure}

	En la Figura \ref{fig:exp1:tiempo_base} se puede observar como la curva generada por el
	tiempo de ejecución puede ser acotada para un $c$ fijo por una función
	$c*N*log(N)$. Para ratificar esta observación se procede dividiendo cada $T$
	por su respectivo $N$ con la intención de generar una curva de forma
	logarítmica.

	\begin{figure}[H]
		\centering
		\caption{}
		\label{fig:tiempo_sobre_n}
		\begin{tikzpicture}
			\begin{axis}[
					title={},
					xlabel={Tamaño de entrada ($N$)},
					ylabel={Tiempo de ejecución (nanosegundos)},
					scaled x ticks=false,
					scaled y ticks=false,
					ymax=100,
					enlargelimits=0.05,
					width=0.5\textwidth,
					height=0.5\textwidth,
					legend pos=south east,
					legend cell align=left,
					xmin=1
				]
				\addplot[color=black] table[x index=0,y index=2]{../exp/kaioKenOutput};
				\addplot[color=red] table[x index=0, y expr={ln(x)*\constante}]{../exp/kaioKenOutput};
				\legend{$\frac{T}{N}$, $c*log(N)$}
			\end{axis}
		\end{tikzpicture}
	\end{figure}

	Con la Figura \ref{fig:tiempo_sobre_n} se corrobora lo previsto, ya que
	efectivamente se puede apreciar cómo con la misma constante $c$ se puede
	acotar por una función $c*log(N)$.

	\begin{figure}[H]
		\centering
		\caption{}
		\label{fig:tiempo_sobre_n_log_n}
		\begin{tikzpicture}
			\begin{axis}[
					title={},
					xlabel={Tamaño de entrada ($N$)},
					ylabel={Tiempo de ejecución (nanosegundos)},
					scaled x ticks=false,
					scaled y ticks=false,
					ymin=0,
					ymax=5,
					enlargelimits=0.05,
					width=0.5\textwidth,
					height=0.5\textwidth,
					legend pos=south east,
					legend cell align=left,
					xmin=1
				]
				\addplot[color=black] table[x index=0,y index=3]{../exp/kaioKenOutput};
				\addplot[color=red] table[x index=0, y expr={\constante}]{../exp/kaioKenOutput};
				\legend{$\frac{T}{N*log(N)}$, $c$}
			\end{axis}
		\end{tikzpicture}
	\end{figure}

	Finalmente, en la Figura \ref{fig:tiempo_sobre_n_log_n} se tiene que
	$\frac{T}{N*log(N)}$ converge a una constante que puede ser acotada por $c =
	\constante$. Este $c$ es el mismo que es utilizado en las figuras
	anteriores.

	Es así como se llega a la conclusión de que efectivamente la complejidad
	temporal de la solución desarrollada coincide con la complejidad teórica estipulada.

	Para reproducir los datos utilizados basta con ejecutar \texttt{KaioKenSolver
	-p}.
