\section{Informe de modificaciones}

A continuación se señalan los cambios realizados para la reentrega del trabajo:

\begin{itemize}
	\item{
		Ejercicio 1
		\begin{itemize}
			\item Se eliminó la decisión de tomar a los dos más rapidos en cada caso. Se reemplazó por abarcar todos los casos posibles de mandar a cada uno de los arqueólogos y caníbales.
			\item Se agregó y modificó el pseudocódigo en solución propuesta.
			\item Se ajustó la cota de complejidad con las modificaciones de código hechas. Y se demostró.
			\item Se agregó la demostración de correctitud.
			\item Se volvió a explicar más detallado los experimentos
			\item Se modificaron los gráficos para que estén más claros.
			\item Se agregó el tercer experimento.
		\end{itemize}
	}
	\item{
		Ejercicio 2
		\begin{itemize}
			\item En el gráfico de la experimentación se ignoraron los primeros
			valores para no tener picos al comienzo. Esto además se aclara en la
			descripción de la sección.
			\item En el gráfico de la experimentación se añadió la cota de complejidad teórica calculada.
			\item Se modificó la explicación de la correctitud (más precisamente la explicación en cuestión de los restos) para que la misma sea más clara y concisa.
		\end{itemize}
	}
	\item{
		Ejercicio 3
		\begin{itemize}
			\item Se modificó la cota de complejidad por $\ord((\prod_{i = 0}^{M-1} K_{i})*(\sum_{i = 0}^{N-1} C_{i}))$
			\item En la experimentación se cambió las escalas de las funciones para que sean posibles de visualizar y comparar.
		\end{itemize}
	}
\end{itemize}
