\section{Apéndice}\label{sec:codigo}

\subsection{Funciones principales de Kaioken}
\begin{lstlisting}
vector< vector<int>> generarPeleas(int n){
  int cantidadPeleas = ceil(log(n)/log(2));
  vector<vector<int>> peleas = 
  vector<vector<int>>(cantidadPeleas, vector<int>(n, 2));
  int saltos = 1;
  for(int i = 0; i < cantidadPeleas; i++){
  for(int j = 0; j < n; j += 2*saltos){
    for(int k = 0; k < saltos; k++){
    if(j + k < n){
      peleas[i][j + k] = 1;
    }
    }
  }
  saltos *= 2;
  }
  return peleas;
}
\end{lstlisting}

\subsection{Funciones principales de Genkidama}
\begin{lstlisting}
vector<int> resolverGenkidama(const int n, const int t, 
	const vector< vector<int>> &coordenadasEnemigos){
  vector<int> solucion;
  int max_y = 0;
  int x_a_cubrir = 0;
  if(coordenadasEnemigos.size() > 1){
    // Primer elemento, si no esta cubierto por el siguiente, lo agrego
    if(coordenadasEnemigos[1][0] + t < coordenadasEnemigos[0][0]){
      solucion.push_back(1);
      // Me guardo la altura para ver si cubre a alguno que este delante
      max_y = coordenadasEnemigos[0][1] + t;
    }
    else{
      x_a_cubrir = coordenadasEnemigos[0][0];
    }
    // Elementos entre el primer y ultimo
    for(unsigned int i = 1; i < coordenadasEnemigos.size() - 1; i++){
      // Si no esta cubierto por un elemento anterior
      if(coordenadasEnemigos[i][1] > max_y){
        // Si el siguiente elemento me cubre
        if(coordenadasEnemigos[i + 1][0] + t >= coordenadasEnemigos[i][0]){
          x_a_cubrir = (x_a_cubrir == 0) ? 
            coordenadasEnemigos[i][0] : x_a_cubrir;
          // Si no cubre al que yo si, entonces me agrego
          if(coordenadasEnemigos[i + 1][0] + t < x_a_cubrir){
            solucion.push_back(i + 1);
            max_y = coordenadasEnemigos[i][1] + t;
            x_a_cubrir = 0;
          }
        }
        else{
          solucion.push_back(i + 1);
          max_y = coordenadasEnemigos[i][1] + t;
          x_a_cubrir = 0;
        }
      }
    }
    // Ultimo elemento
    if(coordenadasEnemigos[n - 1][1] > max_y){
      solucion.push_back(n);
    }
  }
  else{
  // Solucion para unico elemento
  solucion.push_back(1);
  }
  return solucion;
}
\end{lstlisting}

\subsection{Funciones principales de Kamehameha}
\begin{lstlisting}
// Saca los enemigos que destruye de enemigosRestantes y los agrega a
// enemigosDestruidos
vector<unsigned int> destruirEnemigos(vector<unsigned int> &enemigosRestantes, 
	Recta kamehameha){
  vector<unsigned int> destruidos;

  unsigned int i = 0;
  while (i < enemigosRestantes.size()) {
    if (kamehameha.pasaPor(coordenadasEnemigos[enemigosRestantes[i]])){
      destruidos.push_back(enemigosRestantes[i]);
      enemigosRestantes.erase(enemigosRestantes.begin() + i);
    }
    else {
      i++;
    }
  }

  return destruidos;
}

bool resolverKamehamehaRecursivo(vector<unsigned int> enemigos, 
	vector<vector<unsigned int>>& solucion, unsigned int limite) {
  // solucion = vector<vector<unsigned int>>();

  if (enemigos.size() == 0) {
    return true;
  } else if (limite > 0) {
    if (enemigos.size() == 1 || enemigos.size() == 2) {
      solucion.push_back(enemigos);
      return true;
    } else {
      vector<vector<unsigned int> > mejorSubsolucion;
      bool haySubsolucionValida = false;
      vector<Recta> kamehamehasProbados;

      // Todas las posibles combinaciones de Kamehamehas sin contar las 
      // que ya hice al reves
      // Ejemplo: si hice un kamehameha de 0,0 a 1,1 no voy a hacer de 
      // 1,1 a 0,0 porque es lo mismo
      for (unsigned int i = 0; i < enemigos.size() - 1; i++){
        for (unsigned int j = i + 1; j < enemigos.size(); j++){
          Recta kamehameha = Recta(coordenadasEnemigos[enemigos[i]], 
          	coordenadasEnemigos[enemigos[j]]);

          bool repetido = false;
          for (unsigned int k = 0; k < kamehamehasProbados.size(); k++) {
            if (kamehameha == kamehamehasProbados[k]) {
              repetido = true;
              break;
            }
          }

          if (!repetido) {
            kamehamehasProbados.push_back(kamehameha);
            vector<unsigned int> enemigosRestantes = enemigos;
            vector<vector<unsigned int> > subsolucion;
            subsolucion.push_back(destruirEnemigos(enemigosRestantes, 
            	kamehameha));

            if (enemigosRestantes.size() > 0) {
              vector<vector<unsigned int> > subsolucionRecursiva;
              if (resolverKamehamehaRecursivo(enemigosRestantes, 
              	subsolucionRecursiva, limite - 1)) {
                haySubsolucionValida = true;
                subsolucion.insert(subsolucion.end(), 
                	subsolucionRecursiva.begin(), subsolucionRecursiva.end());
                mejorSubsolucion = subsolucion;
                limite = subsolucion.size();
              }
            } else {
              haySubsolucionValida = true;
              mejorSubsolucion = subsolucion;
              limite = 1;
            }
          }
        }
      }

      if (haySubsolucionValida) {
        solucion.insert(solucion.end(), mejorSubsolucion.begin(), 
        	mejorSubsolucion.end());
        return true;
      }
    }
  }

  return false;
}
\end{lstlisting}
